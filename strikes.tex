\chapter{---STRIKES.} \label{STRIKES}

When a body of men stand out for a price which their employers refuse to
give, while this dispute is pending, the position of the workmen is that
of a strike. As strikes are the last resort, as they are always
expensive, and as they engender mutual ill feeling, they should never be
entered into without duly calculating the probabilities of success, nor
until all means of amicably settling, the difference have failed. It
often happens that workmen have no alternative, but either to submit to
a reduction tyrannically enforced, without any reasoning on the matter
being allowed, or to cease from labour. Often has a strike thus been
precipitated, and ruin inflicted on employer and employed, which might
have been averted by a little calm reasoning on the matter. It is the
same when a rise in wages is asked by the men. Both parties are apt to
view each other as enemies, and in this jaundiced view, which prevails
equally on both sides---aggravated by the unconciliatory tone which is
sure to result from such a state of feeling---reasoning on the
subject---as the subject, considering the important results to both
paHies which are then pending, should be reasoned upon---is rendered
impossible. The beginning of strife is like the letting out of water
that might be, at the commencement, easily stopped. But if there be one
thing more than another which, in their turn, both parties in these
circumstances often, to all appearance, agree in throwing aside, it is
the conciliatory spirit which might prevent these ruinous disputes. But
while strikes are always to be deprecated, because they are, for the
time, a state of moral warfare, and, like all sates of hostility,
productive of mutual bitterness---and because they are carried on at a
loss to both parties---we are, notwithstanding, clearly of opinion, from
long experience of their results to journeymen both of success and
defeat, that there is no proper alternative, in certain cases, than the
position of a strike.

The following extract from a very talented article in the
\textit{Builder} of August 11, 1859, which very clearly sets forth the
nature of strikes, as clearly exhibits their rationale:---

\begin{quote}
    \hspace{2em} Suppose a question of cotton and sugar value instead of
    labour value---how does the seller know whether he is selling too
    cheap, except by refusing to sell at all below a certain higher
    rate? If there is but little cotton or sugar, as of course he
    suspects or affects to suspect there to be, he will sell what he
    chooses at his own figure; but if not, he must take the buyer's
    price for it. Now, the fixed price about which the transaction halts
    is the strike of the seller against the buyer---of the supplier
    against the demander---and provides the only practicable means of
    arriving at the fair value.
\end{quote}

Against strikes it has been often urged---

\begin{enumerate}
    \item Their great expense;
    \item That they promote the introduction of machinery, and
    consequently leave the workmen in a tenfold worse position than they
    were before.
\end{enumerate}

1. It is
