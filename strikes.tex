\chapter{---STRIKES.} \label{STRIKES}

When a body of men stand out for a price which their employers refuse to
give, while this dispute is pending, the position of the workmen is that
of a strike. As strikes are the last resort, as they are always
expensive, and as they engender mutual ill feeling, they should never be
entered into without duly calculating the probabilities of success, nor
until all means of amicably settling, the difference have failed. It
often happens that workmen have no alternative, but either to submit to
a reduction tyrannically enforced, without any reasoning on the matter
being allowed, or to cease from labour. Often has a strike thus been
precipitated, and ruin inflicted on employer and employed, which might
have been averted by a little calm reasoning on the matter. It is the
same when a rise in wages is asked by the men. Both parties are apt to
view each other as enemies, and in this jaundiced view, which prevails
equally on both sides---aggravated by the unconciliatory tone which is
sure to result from such a state of feeling---reasoning on the
subject---as the subject, considering the important results to both
paHies which are then pending, should be reasoned upon---is rendered
impossible. The beginning of strife is like the letting out of water
that might be, at the commencement, easily stopped. But if there be one
thing more than another which, in their turn, both parties in these
circumstances often, to all appearance, agree in throwing aside, it is
the conciliatory spirit which might prevent these ruinous disputes. But
while strikes are always to be deprecated, because they are, for the
time, a state of moral warfare, and, like all sates of hostility,
productive of mutual bitterness---and because they are carried on at a
loss to both parties---we are, notwithstanding, clearly of opinion, from
long experience of their results to journeymen both of success and
defeat, that there is no proper alternative, in certain cases, than the
position of a strike.

The following extract from a very talented article in the
\textit{Builder} of August 11, 1859, which very clearly sets forth the
nature of strikes, as clearly exhibits their rationale:---

\begin{quote}
    \hspace{2em} Suppose a question of cotton and sugar value instead of
    labour value---how does the seller know whether he is selling too
    cheap, except by refusing to sell at all below a certain higher
    rate? If there is but little cotton or sugar, as of course he
    suspects or affects to suspect there to be, he will sell what he
    chooses at his own figure; but if not, he must take the buyer's
    price for it. Now, the fixed price about which the transaction halts
    is the strike of the seller against the buyer---of the supplier
    against the demander---and provides the only practicable means of
    arriving at the fair value.
\end{quote}

Against strikes it has been often urged---

\begin{enumerate}
    \item Their great expense;
    \item That they promote the introduction of machinery, and
    consequently leave the workmen in a tenfold worse position than they
    were before.
\end{enumerate}

1. It is admitted that strikes are very expensive. But the expense of
anything must be taken in reference to the gain it is intended to
procure, or the loss it is intended to avert. It should also be viewed
in reference to its result in success or defeat.

The expense of strikes to prevent a reduction of wages, let it be what
it may, is soon equalled by the amount it is intended to take from the
men's wages. A reduction of a penny per hour---we take this sum, because
in our own trade reductions have been attempted which would amount to
it, but any other sum can be taken, as the case may be---in a day often
hours, occurring in a trade of 1,000 men, amounts to £250 a week, or
£13,000 a year.\footnote{At 5 per cent., £13,000 a year represents a
capital of £260,000} This may be called a small trade. In a trade of
10,000 men it amounts to £130,000 a year;\footnote{At 5 per cent.,
130,000 a year represents a capital of £2,600,000} and in a trade of
20,000 men it amounts to £260,000 a year.\footnote{At 5per cent.,
£260,000 a year represents a capital of £5,200,000} Considering the
capital these sums per annum represent, it cannot be surprising that
workmen are willing to incur a very great expense to prevent loss so
enormous.

But suppose the men to be defeated after incurring great expense---the
employers' expenses are sure to be fully equal to those of their men,
besides the possible loss of business. The fact of its being often very
expensive to reduce wages, prevents reductions being attempted which
otherwise would be made without hesitation or scruple. Strikes,
therefore, even in defeat, have a powerful tendency to prevent a future
and further reduction of wages.

Besides, it might as well be urged because in a comparatively short
period millions have been expended in going to law, involving often
total ruin to families, loss of reason, and even suicide, that therefore
no unjust aggressive claim should be opposed, and referred to legal
decision.

With respect to the objection, that strikes promote the introduction of
machinery: at no time is a strike needed, or indeed is there any
stimulus, but the shortening of labour, necessary for its introduction,
whenever it is found to answer. It may be true that ``mules'' and
``double deckers'' were introduced during, or about the time of a
strike; but these, if they saved labour, would have been used if there
had been no strike. The spinning jenny was not invented in consequence
of a strike; neither was the power-loom. No strike existed when the
printing machine was introduced; nor was there any to occasion its
subsequent improvements. Nor has been, or would, the introduction of
machinery in any trade be delayed an hour beyond the time it was found
to answer. ``The ingenuity of our mechanics,'' alluded to in the
``Quarterly Review'' (p. 504) has never needed the impulse of a strike
to call it into action; indeed, the fact has been exactly the contrary.
It has been more apt to employ itself in the making of machines which
were found not to pay, than to require any stimulus, whether of a strike
or otherwise, to call it into action.

Two great mistakes are often committed on both sides in strikes. The
first is, in forgetting that the issue joined is simply to prove which
of the two parties can stand out longest in a bargain---that the dispute
is not contending as in a combat, but simply for the adjustment of a
certain price for labour. It is quite true, that, where a reduction of
wages results from one employer underselling other employers, and who is
seeking to make up his profit out of his men's wages, a sense of wrong
is felt by the men so reduced, similar to that which prompts a man to
repel an injury. And, doubtless, in such a case they think that such an
employer is seeking to extend his trade by taking what does not belong
to him, but to themselves. And, on the other hand, where a rise in wages
is asked, which the employer feels will reduce his profits, and much
narrow his operations, he also will feel the same. Nevertheless, to
bring such a contest to an issue at the least possible cost to both
parties, and the least injury to the trade itself, such feelings ought
to be laid aside, and the dispute conducted according to what it really
is, simply a bargain, in which the employer is the buyer, and the
employed is the seller. Both are standing out against each other's
price. If the seller in this case can do without selling, and the buyer
can get no labour elsewhere, the workman is sure to win. If the
contrary, the employer. It is indispensable, therefore, as the dispute
cannot be carried on without great loss to both, that each party should
be well informed of every circumstance connected with that their true
position, both at the outset and from day to day as the dispute
proceeds. This, however, is not always done, and is the other mistake
often committed. In the builders' strike and ``lock-out,'' if the
masters had at all studied their position as to the number of men likely
to be available in the precipitated strike at Messrs. Trollope's, they
would have found, if their own statements are to be depended on, that
there was no occasion whatever for their lock-out, with its great
expense to themselves, and suffering to their men. It was, as they
expressed it, ``to continue for a month, or until the Messrs. Trollope
and Sons were enabled to resume their operations.''

This firm, by its own statement, was enabled to resume operations in a
fortnight, without having gained over a single man that had been locked
out, either of their own late workmen or those of any other of the
combined employers; and it was the same at the subsequent period, when
they announced that they had obtained their full complement of men. And,
as the lock-out, therefore, did not achieve this, it was indisputably
not needed at all. Their loss of £130,000, as announced in the
\textit{Weekly Dispatch} of January 1, 1860, which, from circumstances
not necessary to allude to here, is unquestionably good authority, has
therefore been incurred without the smallest occasion for it. Many
blunders have no doubt been made by workmen in their strikes, but none
so great as this, because so entirely without real necessity.

Both parties, however, are too often blinded by passion, to see clearly
their way, and being angry, the contest is transformed from what it
really is, merely of a bargain, into that of a combat. Under the
influence of this excitement, the leaders of both parties are too often
more intent upon setting forth the injuries which their respective
audiences imagine they shall sustain, if they abate the least of their
original views; or, what is more mischievous, they feel it a duty to
urge that they cannot without \textit{disgrace} modify, for the purpose
of arrangement, the several points in dispute, forgetting all the while
that it is only a bargain for which they are contending, in which it is
the special business and duty of the seller, if he finds he cannot get
his price, and the market is being, or likely to be, supplied, to take
the next in value, and in no case to stand out, if he sees the market
being supplied without him. Among dealers, he who loses by standing out
too long is always an object of derision. This mistake is too often made
in a strike. The dogged valour which has so often enabled the sons of
toil to conquer in battle is not unfrequently the cause of total defeat
in a strike, when, with a little address, considerable advantage might
have been gained, or loss prevented. Such address has been considered
cowardly and disgraceful, and the affair has ended with the loss of
everything. On the other hand, the employers, under the influence of the
same feelings and mistake, have often been led to refuse all adjustment
or arbitration, and thereby have led, by protracting the struggle, to a
much greater loss than their victory has achieved gain; and in some
cases, as in the master-builders' lock-out, to immense loss, without any
substantial gain at all, unless the gratification of their anger be so
considered.

But, after all, the true position of employer and employed is that of
amity. They are each, notwithstanding these occasional disagreements,
the truest friends of the other, and neither can inflict an injury on
the other without its recoiling on himself. Capital and Labour should go
hand in hand. Experience has amply proved that the Capitalist cannot
injure the Labourer, or the Labourer the Capitalist, without each
inflicting injury, and perhaps ruin, upon themselves.

\section{THE ``EDINBURGH REVIEW.''}

Many of the charges by this reviewer have already been answered. It is
necessary, however, briefly to notice the article. It is written,
apparently, \textit{to order}---confessedly, in profound ignorance of
the subject. This ignorance may be real or feigned---feigned, in order
to induce a belief that Trades' Unions were secret societies, similar to
those which ``honeycomb continental nations,'' abounding in ``spies,''
with ``forced dumbness,'' and ``\textit{surveillance}'' and ``distrust
of neighbours on the right hand and left,'' of which no information
could be with certainty obtained. This, indeed, he largely insinuates,
which is turning his ignorance, if real, to very good account. And it
must be confessed, if this ignorance be feigned, that the acting, if it
be not altogether successful, is very effective. Knowing nothing, or
being obliged, from the course he has adopted, to pretend to know
nothing of the subject, and finding that Adam Smith and McCulloch were
dead against him, and Stuart Mill, who was formerly so, having only
recently modified his opinion and even now enjoining that there should
be no encouragement given to the multiplication of labourers, because in
the opinion of ``farmers,'' ``Boards of Guardians,'' and others of the
employer class, the checking of such multiplication would make them
``too independent,'' and therefore, besides in other respects, not a
presentable authority for his intended article, he with admirable
adroitness tells his readers at the commencement that he is ``not going
to preach political economy.''

Nor was the information respecting Trades' Unions that could be obtained
from ``Blue Books,'' namely, the Parliamentary Enquiry on Combinations of
1838, and the Report of the Hand-loom Weavers' Commission suitable for
his purpose; the last, though quoted by him at the end, not being
applicable to any extent, as the hand-loom weavers were not Trades'
Unionists, which is one reason given in the before-mentioned Enquiry of
1838 (Questions 1418--19) why their wages were so wretchedly low; while
the evidence in the same Enquiry (of 1838), which is divided into two
parts, the first containing chiefly the evidence on the cotton-spinners'
strike of 1837, and the second the evidence on Irish Trades' Unions, is
too conflicting to be used by him, who only intended to see the
employers' side of the question. Indeed, it is so conflicting that the
Committee made no report upon it. Besides, the first of these contained
the opinions of Sir Archibald Alison on Trades' Unions, whom no one will
suspect of having an undue leaning towards them. Apart from intimidation
and violence which we are as much against as he could possibly be he
expresses himself several times favourable to trades' combinations, in
terms similar to the following extract. He is speaking of the
CottonSpinners' Union of Glasgow:---

\begin{quote}
    \hspace{2em} ``Their combination, I think, might have a very
    beneficial effect in their own favour, if it was limited to merely
    legal acts. I think, for example, that it is a very good thing for
    the cotton-spinners, as well as every other class of labour, to
    combine, because it enables numbers, to a certain degree, to
    compensate and to enter with equality into the lists with capital;
    and therefore I think that combinations are essential to support the
    rights of labour in the competition with capital.'' (Question 1956.)
\end{quote}

In this state of things, where it was so essential to shut himself out
from using, if he possessed it, any knowledge of the subject, there was
nothing for it but to follow the course pursued by Lord Chesterfield
when he brought the Bill into the House of Lords for reforming the
Calendar. His lordship tells his son, in one of his Letters, that having
to bring in this Bill; knowing nothing of astronomy, and consequently
unable to explain the reasons for the change; he did what he believed
his audience would be better pleased with, namely, instead of urging the
reason for the alteration from the science---which, he further said, he
did not believe his audience would have understood if he had been able
to give such explanation---he gave a history of the changes made in the
Calendar down to the then present time, taking care to flatter the
prejudices of his hearers by telling them what he thought they would
like to hear, by which they were so delighted that they gave him credit
for his profound knowledge of the science; while another noble lord, who
well understood astronomy, and who handled the subject with the learning
of a philosopher, was heard with impatience. With this example of the
advantage of being ignorant of the subject upon which he was about to
write, which he betters, by turning his real or assumed ignorance to the
account described:---he proceeds to give a history of alleged misdeeds
of workmen when combined---of many of their strikes, in which they are
described to be everything that is unreasonable, unjust, tyrannical, and
absurd, in the hope, as already in part quoted, that his readers will
believe of Trades' Unions that---

\begin{quote}
    \hspace{2em} Their aim and object is, in every case which we have
    been enabled to investigate, to \textit{stint} the action of
    superior physical strength, moral industry, or intelligent skill; to
    depress the best workman in order to protect the inferior workman
    from competition; to create barriers which no Society-man can
    surmount, and which few non- Society men dare to assail; and, in
    short, to apply all the fallacies of the protective system to
    labour. Such a system injures, first, the individual, whom it robs
    of a free market for his labour; secondly, the class of manufactures
    to which he belongs, by increasing the cost and diminishing the
    efficiency of the workmen; and, lastly, the nation at large, by
    curtailing the productive power, and consequently the wealth of the
    community.---P. 529.
\end{quote}

The only result of his ``investigation,'' as exhibited by himself, has
been the total exclusion of every fact and circumstance that would
justify, excuse, or extenuate the proceedings of the men, which, indeed,
appears to have been the only object he had in view in setting himself
to investigate the matter.

We have already shown the fallacies of the assertions, upon which this
confused argument rests, itself consisting principally of mere
assertion. The only part we have not alluded to that might be construed
into an application of the ``fallacies of the protective system to
labour,'' is the insisting, as Trades' Unions do, that all their members
should serve an apprenticeship, generally of seven years, to their
trade, and of sometimes endeavouring to limit the number of apprentices.
This last, however, is not now the general practice, but the exception.
It is always a needy or grasping employer who takes an inordinate number
of apprentices, and he never cares a straw for ``the nation at large,''
or ``the wealth of the community;'' he only aims, in thus using ``its
productive power,'' to enrich himself, no matter at whose expense.

The Act of Elizabeth, requiring an apprenticeship of seven years before
exercising a trade, has been repealed nearly fifty years, after having
been in force for two centuries and a half The ``Quarterly Review,'' in
its late article against Trades' Unions, states that this Act
``unquestionably exercised an important influence on English industry.
It stigmatised and punished the idle and the vagabond, directed the mass
of the people to manual occupation as affording the best means of
independent subsistence; and being acted on steadily from generation to
generation, it gradually educated a nation of skilled artizans.''---P.
480. And although the Act has been repealed so long, the practice has
not fallen into disuse. In some trades there is, probably, less to be
learned than in others; but there are very few trades in which long
experience and practice are not necessary, before the workman is able to
compete with those who are considered valuable to the employer. This
ability must be acquired at some period, and what more proper than that
in which the individual passes from youth to manhood---a time in which,
of all others, a facility of hand, and dexterity in performing difficult
processes, is the easiest acquired. This, then, being the most proper
period, and at the same time this also embracing the period when the
passions develope themselves, and are the most ungovernable, does it not
follow irresistibly that, if ever in a man's life he need the
experienced direction and restraint of his elders, it is this period? An
apprenticeship of seven years, from the age of fourteen to twenty-one,
whether considered necessary to learning a trade, as the best period for
acquiring it, or as conducing to the future welfare of the apprentice
himself, points itself out therefore as absolutely necessary. And this
is the reason why the repeal of the Act has made so little difference in
the practice, and why Trades' Unions, generally speaking, insist that
all their members should have served ``their time as apprentices.'' The
members are most of them fathers, and they know by experience the
beneficial effects of the practice, and the mischief that attends the
letting boys ``do as they like'' at this critical period. But we shall
be told this is no reason why the ``productive power, and consequently
the wealth of the community,'' should be ``curtailed.'' True. And the
same reason exists why the ``master villany of the earth, which includes
every other villany,'' and deadly sin---slavery, should perpetually
continue; for how else can ``the wealth of the community'' be so well
kept up by supplies of cheap cotton and sugar?

By the ``nation at large'' we suppose is meant what is termed the
``public.'' The public cares very little how things are produced, so it
gets them cheap, and the cheaper they become the more the public likes
it. The public, however, always respects those who so take care of
themselves as to prevent their suffering wrong. All the reasoning in the
world, nor all the admonitions of virtue, benevolence, or religion,
would prevent the public from preferring slave-grown sugar or
cotton---all the horrors of the slave trade notwithstanding---if it were
cheaper and as good as either grown by free labour. But, in justice to
the public it must be admitted, that it is also true, that, in the event
of men refusing to be slaves---supposing such a thing were to
happen---it would receive with a shout of mingled execration and
derision the wailing of those who complained that commodities were dear
because men were no longer enslaved to make them cheap. The reviewer,
therefore, has no occasion to make himself uneasy about the ``nation at
large;'' it never scruples to take care of itself, or fails to respect
those who do the same.

In giving his history of the alleged misdeeds of Trades' Unions, he has
not bettered his example. Nothing was ever overdone by Lord
Chesterfield. Indeed, on second thoughts we begin to doubt whether he
ever read the works of that nobleman. He proves too much by his
exaggeration; that is, if a jumble of alleged facts, disjointed from all
proof that they of necessity belong to Trades' Unions, can be said to
prove anything. And it is superfluous to say that he who proves too much
proves nothing. He wishes it to be believed that these misdeeds were
done by Trades' Unionists in and by their Union. If even they were all
true, to make them prove anything against Trade Societies it is
indispensable that he should also show of each and all that Trades'
Unions could not exist without these or similar enormities. This he does
not attempt to do; indeed he has entirely shut himself out from doing
so, by his real or assumed ignorance of their action---by his own
representation that they are secret, and that therefore nothing can be
known about them. He seems to have forgotten that his real or pretended
ignorance of their nature and action, so convenient for dark
insinuation, makes him incompetent even to argue the subject, much less
to prove anything, and consequently worthless as an authority. We have
already shown the utter fallacy of his insinuation, that Trade Societies
are secret, irresponsible in their government, or tyrannical; but that,
on the contrary, they are instituted for the purpose of correcting the
disadvantage inherent to the position of the working man when he
bargains \textit{singly} for the sale of his labour, namely, of being
compelled by his immediate necessities to take a price for it below its
fair exchangeable value. To assert that they, in any given instances,
have been turned to a bad purpose, is only to say of them what may with
truth be said of everything human.

