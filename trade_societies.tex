\chapter{---TRADE SOCIETIES.} \label{TRADE_SOCIETIES}

The object intended is carried out by providing a fund for the support
of its members when out of employ, for a certain number of weeks in the
year. This is the usual and regular way in which the labour of the
members of a Trade Society is protected, that the man's present
necessities may not compel him to take less than the wages which the
demand and supply of labour in the trade have previously adjured;
strikes, which we shall hereafter notice, being only resorted to on
extraordinary, and, generally speaking, most unusual occasions.

As the object of combinations is thus to correct the disadvantage of
position as bargainers with their employers, and as the very element of
their existence is public spirit, being held together only by a
principle of honour to support each other for a perfectly legitimate
purpose, they always comprise not only the best workmen, but best men in
a moral sense that are to be found in the trade. This is stated, simply
because it is a fact, which has been acknowledged by those who are
inimical to trade combinations. As the social position of workmen
depends entirely upon the wages they obtain, it is felt by the majority
of them to be a sacred duty to adopt this means of protecting their
wages. There is no state so abject, and no tyranny so hard to be
endured, as that which results from the seller being completely at the
mercy of the buyer. It is to avoid this state so detrimental to
themselves, and so advantageous to the employers ---at least so thought
by many of them, ---that working men combine, and that their
combinations, according to the ``Edinburgh'' reviewer, contain ``about
600,000 members, commanding a fund of £300,000.'' How this fund is
disposed of we shall hereafter allude to. This is the reason, at which
neither he nor any man need ``for the twentieth time wonder,'' that
induces ``intelligent workmen'' to form themselves into Trades' Unions;
and not, as stated by the same reviewer, after about four pages of vague
denunciation and foolish wonder, to ``intimidate,'' ``to stint the
action of superior physical strength, moral industry, or intelligent
skill,'' or ``to apply the fallacies of the protective system to
labour;'' their object, on the contrary, being to ensure the freedom of
the principle of EXCHANGE with regard to labour, by, as before observed,
putting the workman in something like an equal position in the bargain
for sale of his labour with his employer. If in doing this they
sometimes---err as who does not?---such error does not invalidate the
propriety of such a course.

The same reviewer, in a page which, though placed at the commencement,
seems to have been written as a kind of summing-up of the whole, asks
his readers: ``Can it be possible, then, that in Great Britain, too,
there is an \textit{imperium} in \textit{imperio} as oppressive to those
whom it comprehends as any secret society on the Continent, enforcing
mysterious laws and arbitrary obligations by the hands of irresponsible
authorities, in defiance of the great natural laws of human society?
Recent events have prepared our readers to take some interest in the
inquiry whether the fact is so or not. When they have learned as much as
can be known of the Trades' Unions of England, they may judge for
themselves whether any existing government in Europe from Constantinople
to St. Petersburg or Paris would venture to exercise so stringent a rule
over its subjects as a large proportion of our working men submit to
from men of their own order. Under the system of irresponsible
government, which strives to subject the labour market to its dominion,
the characteristic freedom of the English citizen is lost. Hundreds of
thousands of our countrymen are unable to say that each man's house is
his castle; that his labour and its rewards are his own property, and
that `live and let live' is the rule of the society in which he
dwells.'' ---P. 528.

The sum and substance of the above, divested of its gloss, is that
Trades' Unions are, 1st, secret; 2nd, irresponsible; and 3rd, coercive.

\section{---SECRET.}

With respect to the ``mysterious laws and arbitrary obligations,'' we
have only to say that the proceedings of Trades' Unions are no more
secret than those of any other societies, to which working men and
others belong. When the reviewer talks of ``Continental nations from the
Mediterranean to the Volga, and from the Black Sea to the Channel, being
honeycombed with secret societies,'' and insinuates that it is a mistake
in Englishmen to congratulate themselves that everything with them is
``open and above board,'' and that they have no ``spies,'' ``enforced
dumbness,'' and ``\textit{surveillance},'' while Trades' Unions exist
among them; he insinuates what is false. The insinuation, for he does
not venture to state it openly, without doubt, gives strength to the
alleged tyranny, but it is untrue. That the proceedings of Trades'
Unions have been unknown is true, but they have only been unknown
because they have been considered by the general public, and therefore
by the press, too obscure and insignificant to deserve attention.
Nothing would serve the cause of the working classes more than a
continual ventilation of their proceedings by the public press; it
would tend to procure for them a greater meed of justice than they have
hitherto found, and, what is better, it would tend greatly to preserve
them from error. There is not a more salutary check to injury from your
own proceedings, as well as from the proceedings of your opponents, than
publicity through the press. This, however, has undergone a change for
the better, during the last few years. In the ``builders' dispute,'' as
far as the men's proceedings are concerned, it has ceased altogether;
the secresy being entirely on the side of the Masters' Association.

We have now before us 227 quarto pages, double columns, for the most
part in the \textit{small} newspaper type, of the reports of the
proceedings of the men, leading articles, \&c., on this dispute, cut
from the newspapers, chiefly from the \textit{Times, while that paper
reported them}, and when it ceased to do so, from the \textit{Morning
Advertiser}, and other papers. This, in the usual type (not the largest),
would make a quarto volume of 671 pages. Where, then, is the secresy?
The masters' proceedings, in their Association, have been secret, but
not the men's, in their ``Trades' Union.''

There are no ``secret societies'' as ``Trades' Unions.'' The attempt to
make them so, in 1834, was utterly abortive, from the refusal of the
trades generally to become so. When it -was illegal to combine, they
were necessarily secret, not from any inclination, but from the force of
the adverse law, which compelled secresy. This enforced secresy
generated its usual train of evils, and in some cases the administration
of oaths to members. When the laws against combinations were repealed,
these practices, and the secresy out of which they grew, ceased, which
ought to have been known to the reviewer, before he attempted to write
on the subject.

The ordinary proceedings of a Trade Society are of course too
uninteresting for public attention. But in disputes, which in the case
of the builders, have occupied so much of the attention of the public,
whose fault has it ever been that they have not obtained the same
publicity? Certainly not the men's. During the strike of the masons at
the building of the two Houses of Parliament in 1841, the greatest
difficulty was experienced by the men in getting their statements into
the newspapers; in fact, they were obliged to resort to placards and
printed bills, because the newspaper press refused to publish them. The
masters found immediate publication in the first-class newspapers for
their statements, but the men could not. The present writer had the
greatest difficulty in getting a report of an investigation into a case
alleged by the men and denied by the masters, relative to this dispute,
into a popular newspaper, that now would publish such a report
immediately. He with his colleague had to wait upon the editor, who had
previously refused insertion; and at last it was inserted---as he
believes---through the intervention of one who was influential in the
management of the paper. We go into this detail to show that it has
never been the fault of Trades' Unions that their proceedings have been
unpublished to the world. But it is said the Committees are secret. It
is true their proceedings are not published, but does that constitute a
secret Committee? If so, then all meetings of men appointed to transact
business, from the Cabinet Council downwards, including all railway and
parochial Committees, are secret conclaves. But these are not secret,
because the names are known of those composing them; the first-named to
all the nation, and the rest to all whom they concern. And, in like
manner, the names of those who form Trade Committees are equally known
to those who appoint them; and, therefore, are not secret.

\section{---IRRESPONSIBLE.}

The ``irresponsible authorities,'' by whom we suppose are meant the
President, Committee-men, and Secretary, of each and every Trades'
Union---if the reviewer was so ignorant of such matters as not to know
it before---these are invariably elected by the Trades' Union to which
they belong, to which they are responsible for what they do during their
term of office, and who are not slow to find fault and censure them if
they do anything displeasing to the body they represent. If he did not
know this, the commonest sense of propriety and regard for truth ought
to have restrained him from writing on the subject until he had obtained
this information. But then there could have been no ``slashing article''
the dull prose of truth would have prevented it. As the responsibility
of the ``authorities'' of Trade Societies will be further shown as we
proceed, there is no need here of saying more.

As to the ``characteristic freedom of the English artizan being lost''
by this ``system of irresponsible government,'' the reviewer
unconsciously, from his ignorance of the subject, has, by the
\textit{reductio ad absurdum}, refuted himself. He describes a state of
things which has no existence. We should no more have believed this
passage about working men, not knowing whether their ``labour and its
rewards were their own property,'' and the rest of it, applied to
ourselves as workmen, had we not been told, than we should have known
that the French writer meant Shakspere when he called him the ``divine
Williams.'' We have shown this passage to several of the ``hundreds and
thousands of our countrymen'' here described, who have one and all, with
much laughter, in the most expressive vernacular, and with vigorous
expletives, denounced the reviewer as the ``biggest of the unwise,'' and
the greatest of the perverters of truth, that they ever heard of. When
any matter, like the present, has reached its last point of absurdity,
there is no more to be said. We therefore take leave of it, especially
as our object is not to criticise the ignorance of the reviewer, but to
state the truth; and, having disposed of the secresy and
irresponsibility---two such powerful adjuncts to the alleged coercion
``which strives to subject the labour market to its dominion''---we
will now proceed to consider the alleged tyranny or coercion itself.

\section{---COERCIVE.}

As the ``coercion,'' apart from acts of intimidation and violence, said
to be exercised by Trades' Unions, has been the theme of accusation,
invective, censure, or expostulation, according to the mood of the
speaker or writer, with the unanimity of a chorus, from Lord Brougham
down to Mr. Buxton, we hope to be excused if we enter somewhat largely
into the subject. It is not only alleged to indnce men to belong to the
Union, but to exist everywhere, in all that Trades' Societies do, or
intend to do. Now, it is the tyranny of the majority over the minority;
then, it is the tyranny of the leaders; or, again, a general undefined
tyranny which pervades the whole body; but always tyranny and coercion.
According to the ``Edinburgh Review'' the tyranny is general, undefined,
but pervading everywhere. The \textit{Times}, Dec. 9, 1859, gives it to
the leaders in the following terms:---

\begin{quote}
    % 简单粗暴第在引用段落中实现缩进……
    \hspace{2em} They strike at the bidding of men who deem it necessary
    to assert a control over the destinies of trade by making a terrible
    example of employers and employed as often as their secret and
    imperious commands are disobeyed. Nor does their fault consist in
    taking too much, but rather in taking too little, interest in their
    own affairs---in permitting themselves to be made mere tools and
    instruments of the will of others in matters in which it is, above
    all things, incumbent on them to judge for themselves, and in
    surrendering their natural liberty, tamely and ingloriously, into
    the hands of an ignorant and inquisitorial despotism.
\end{quote}

Mr. Black, taking up the subject from the ``Edinburgh Review,'' the said
reviewer having used the word ``excommunication,'' says:---

\begin{quote}
    \hspace{2em} But it is not by illegal violence alone that personal
    liberty may be encroached upon; there are other means of
    intimidation that may be as effectual, or even more so. The Pope, by
    the threat of excommunication, was able to control the most powerful
    sovereigns. A similar threat is one of the most effective
    instruments in the hands of the committees of the Unions. When they
    order a strike, he would be a bold man who would preach the right of
    private judgment, and act upon it in opposition to the commands of
    the secret committee; he would not only be shunned by all his
    companions, but reproached and denounced as a snob, a nobstick, as
    unfair, and insulted in other opprobrious terms. By cajolery,
    molestation, and threatening, the combined leaders strike a general
    terror into the whole community.---P. 21.
\end{quote}

The ``Quarterly'' reviewer, having more sense, does not harp so much on
this string.

To Mr. Black is due the merit of being definite; he not only states what
is done, but the manner of execution. And we have no doubt but that he
has, in the case he has supposed, stated the true impression on his
mind, as derived from the ``Edinburgh Review,'' and the statement of
various speakers and writers on the same side, who are considered
authorities on the subject. We will, therefore, analyse it. The
``excommunication'' is relative to a supposed strike, and he assumes it
to be complete and effectual. Now, how can a man be ejected or driven
from a community, with ``reproaches and denunciations,'' ``shunned even
by his friends,'' unless by the active co-operation of the community
itself? and how could such co-operation exist by the community, unless
such `` excommunication'' had its full and hearty approval? No cajolery
can make men, and especially working men, do what they have no mind to,
and, if the community are willing to eject one of their body so
vigorously as here described, with ``insult'' and ``opprobrious'' terms,
there needs no ``cajolery'' to induce them to do so. As to the
committee, who are as one to fifty, or more, attempting to intimidate
the general body to do such a thing by threats, it is too preposterous
for argument. Mr. Black here describes the tyranny of the majority over
the minority; and not, as he states, the tyranny of its leaders over the
general body. His own case proves that he is entirely mistaken in this
respect. It may be, in such a case, that the leaders are compelled to
act in obedience to the general voice, as has been the fact, hundreds of
times, with more important bodies than Trade Societies. So far from the
fact being, as the Times and Mr. Black suppose, that the Committees of
Trades' Unions are the agitators and leaders in strikes; it oftener
happens than not that they prevent them, by soothing down the angry
feelings excited by what the members of their Societies have considered
unjust treatment, from foremen or employers. There, however, exists some
cause for this error. When men have been asked, by employers, why they
have struck or have refused to work for a certain price, they too often,
from the fear of compromising themselves, lay it all to ``their mates,''
or to ``the Society,'' or to its ``Committee,'' when perhaps some of
those very men have been foremost in setting forth to the meeting of
their general body, which unanimously agreed the strike should take
place, the absolute necessity of so doing. But, although this cause for
error does sometimes exist, it is not the less a great mistake to
suppose that the working class are the mere ``tools and instruments of
the will of others,'' or that they ``surrender their natural liberty''
into the hands of their leaders. Those leaders, as before observed, are
in all cases elected by the general body, to carry out its views, and
therefore cannot be secret. Nor can they do anything against the will of
that body. If they were to attempt to do so, they would soon cease to be
its leaders, and, consequently, could not exercise ``despotic'' rule
over it.

But it may be asserted that, granting this, the rule of Trades' Unions
is nevertheless the tyranny of the majority over the minority. To which
we reply, where is it that this kind of tyranny does not exist? Mr.
Black tells us that there are no ``Trades' Unions'' or ``Strikes'' in
the United States of America, and we believe he is nearly correct.
Where, according to M. de Tocqueville, is the tyranny of the majority
over the minority greater than in those states? To come nearer home,
what is the ``world's dread laugh'' but the tyranny of the majority?
What is ``public opinion,'' so potent, and so fatal to those who stand
in opposition to it, but the voice of the majority against the minority?
who never fail to denounce it as tyranny; and who generally get laughed
at for their folly by the same ``public.'' If this be a crime in Trades'
Unions, it is a crime of which the \textit{élite} of English society are
equally guilty; and not only the \textit{élite} but every grade
downwards from it to the working classes. Out of many instances of the
way in which it sometimes acts that might be given, we only need mention
the following:---Mr. Burke, after his expression of opinion on the
French Revolution, was, to use his own words, ``excommunicated by his
party,'' lost his seat in Parliament, and for his whole life after was
reduced to political insignificance. Lord Chancellor Yorke, who took
office at the earnest request of the King (George III.) from being
supposed by such taking of office to have betrayed his party, was
proscribed by that party; which he took so much to heart, that he lost
his reason, and died by his own hand.

Sir Robert Peel was considered, by the party to which he had for years
belonged, to have betrayed it, though it was all the while known that he
had a perfect right to change his opinion; and the event, during his
life, proved that it was most beneficial to the nation that he had done
so---yet with his party he lost caste; and though he had the approbation
of the general public, his public life was embittered to the end---even
when the aggregate majority was with him---by the most bitter attacks
that ever assailed a public man; against which the approbation of the
general majority was of no avail, such approbation being expressly
because he had shattered to pieces the favourite policy of his former
party; or, to use its owm words, betrayed it.

Messrs. Bright and Cobden, during the late war with Russia, thought
proper on the peace question to express what was known previously to be
their sincere convictions, and which they had an undoubted right to
express, but which were opposed to what was at that time considered to
be the public weal; in short, to the almost wishing success to the enemy
with whom the country was at war. All their past services were
forgotten, they both lost their seats in Parliament, and, during that
war, their influence as public men; and would again under the same
circumstances suffer the same loss, supposing they should, which is very
doubtful, ever regain the confidence which the public formerly reposed
in them. Pausing for a moment on this phase of the subject: do either of
these reviewers, or any one else, require to be told, that a state of
combination, whether it be of a political party or a Trades' Union,
supposes a united action which is inconsistent with dissent?---or that
there are men everywhere, who, after the whole of the fellows---
themselves included---have agreed to a certain movement, have frustrated
it by an opposite course of action? Have not many such been denounced in
both reviews as having ``betrayed their party,'' and ``their cause?''
Without doubt such men have a right to do as they like, and to be
protected from violence and insult; but it is too much to expect that
their fellows should receive them with affection or regard. Do what you
will, talk how you like, such men will always be considered by their
party, whether it be of the highest or lowest sections of the community,
as infamous, and treated accordingly by the majority, whose cause they
have betrayed. If this be ``tyranny,'' then it will never be purged from
the working or any other class of society, while men are human. When, if
ever, they become angelic, it may be different; but then there will be
no traitors.

But to return. What is the end and aim of the ``Quarterly and Edinburgh
Reviews'' as far as politics are concerned, but to excite the public
opinion against their opponents?---and the greatest success they hope to
achieve is to obtain the general voice against those whom they denounce,
or, in short, to create this ``coercion'' or ``tyranny'' of the majority
against them. What is the ``Edinburgh Review'' now doing, but, Tight or
wrong, endeavouring to excite public opinion against Trades' Unions?
What is tbe object of Mr. Black's lecture and Mr. Buxton's letter---both
compiled from the two reviews in question---but to do the same thing?
But we shall be told they all think they are doing right. Do they? We
should be glad to be informed who does not when thus employed? But, if
they think they are doing right in thus invoking public opinion, can
there be a more profound acknowledgment of its power, and of submission
to its voice, than such invocation? It is the acknowledgment of a
superior power by appeals to it. If, then, all in their turn acknowledge
its power by invoking it, as a rule to which all must submit, how can
it be a degrading tyranny when it exercises its force among working men?

But what do they want public opinion to do, when all in their turn
invoke it, but to bring its force to bear against their opponents?---to
bring, if successful, what those opponents would most surely call this
tyranny of the majority against them, with all its irresistible forms of
coercion? And this is deemed by all parties not only legitimate and
proper, but in the highest degree commendable. Is it, then, only when
this formidable power produces the same result among the working
classes, that, in the opinion of this reviewer and the rest, it is
tyranny and coercion? Take the question as it is put by Lord St.
Leonards, in his letter to the \textit{Times} of December 13, 1859, and
the same result will follow---namely, that it is by the opinion of the
majority that Trades' Unions exist---who are said in these unions to
coerce all the rest---that is, the minority. His lordship, as far as we
can understand him, wishes to pat an end to Trades' Unions, and of
course describes their action as most unwise and tyrannical. He says
that, ``If a law were to be passed, placing the free workmen under such
a rule, we might expect an insurrection; and yet vast masses voluntarily
place themselves under a yoke which they would resist in the shape of
authority with all their energies.'' What are the vast masses here
alluded to, but the majority f And if they elect to place themselves
under what his lordship is pleased to term a ``yoke,'' he may rest
assured that they have their own good reasons for doing so, and do not
require, however well intentioned, the commiseration of anyone. Here, as
in the case put by Mr. Black, to suppose that the ``mandates'' of the
committees, \&c., would for an instant be regarded by the general body
comprising the Trade Society, unless those ``mandates'' had previously
had their sanction and concurrence, is too ridiculous for argument.
These officers and committees, once more, do not make laws for the
general body, they are made by the general body itself in all cases; and
if it assigns a discretionary power to its officers, it is because
it---namely, the body at large---think that the best course to pursue.
It seems to be forgotten that working men are not so ``handy'' with the
pen as lawyers' clerks, and therefore the real fact---namely, that it is
better to leave to the discretion of the committee the dealing with
cases that may arise, than to make rules concerning supposed cases which
might, in practice, be wholly inapplicable---is not believed. But,
instead, alarm is taken, or pretended to be taken, at the ``
irresponsible power'' of the committees. Never was alarm more
groundless. Only let any committee or officer act in opposition to the
wish of the general body, and, as before said, they would speedily find
themselves relieved from ``the cares of office.'' But why is all this
care for the ``vast masses?'' They have never, as we have heard,
complained of the ``yoke'' under which they are said to be bowed down by
an ``irresponsible and secret tyranny.'' No complaint of theirs has
evoked all this sympathy. Loud have been their complaints of the tyranny
of the master builders' ``lock-out,'' and large have been their
contributions to resist it. But not a word against this ``yoke,'' so
feelingly alluded to by his lordship. What is the reason of this? It
cannot be because they do not complain when they see a reason:---the
masters' ``dlock-out,'' for instance? What can be the cause of this
silence? We will tell his lordship and the chorus of commiserating
reviewers, lecturers, letter-writers, and others. It is because they,
that is, the ``vast masses,'' believe that by their organisation---``yoke''
his lordship is pleased to term it---they obtain a fair exchange for their
labour, which it would be impossible to secure singly as individuals:
in other words, that they obtain higher wages in combination than they
could if they were not; and they are right. Take the engineering trade,
in illustration. In Glasgow and Newcastle the custom is, not to belong
to the Engineers' Society; in those two places the wages are low
compared with their rate in those districts where the custom is to
belong to it. And even in those two places the men who do belong to the
society obtain a higher rate of wages than the non-society men. Such
instances might be multiplied in every trade. It is, therefore, a matter
of surprise that his lordship, knowing, as he does, what men will
endure, what risks they will undergo, for increased remuneration, which,
as he knows also, they would do for no living man,---that he should think
it extraordinary that the working classes should ``voluntarily place
themselves under'' this ``yoke,'' even at his own estimate of it, to
secure the same thing for themselves.

But all this points with increased force to the fact, that it is the
majority, the ``vast masses'' who determine to combine, and who in
combination decide for themselves what course they shall pursue.

Does his lordship, in alluding to ``tens of thousands'' of men being
suddenly deprived of work---as if they had been struck by paralysis---
allude to the builders' dispute? If so, he should remember that Avas the
act of the master builders by their ``lock-out.'' To this, however, we
shall refer elsewhere. Having thus shown that there is no ``coercion''
necessary to induce men to combine, but that of the ``public opinion'' of
the class who combine; and that trade combinations are neither secret,
irresponsible, nor their rule dependent on the will of their leaders; it
remains only to notice the charges that workmen ``coerce'' all in their
trade to belong to the trade society

Whatever may have been the practice formerly, Trades' Unions are not now
generally charged with acts of physical violence---acts which we
deprecate and abhor. On this point we are in perfect accord with both
the `` Quarterly'' and ``Edinburgh'' reviewers, and those who have
compiled speeches, lectures, and letters therefrom. Wherever such acts
occur, they are a ``heavy blow and great discouragement'' to trade
societies. Leaving these, as things which happily have now passed away,
we need only say in reference to this, that there is no other coercion
than that of the public opinion of the class to which workmen belong.
Although we know of many instances to the contrary in our own trade, to
a certain extent men in a society shop \textit{are} expected to belong
to the society, for the very natural reason, that, as it is believed
wages are kept from falling below the demand and supply rate, by the
contributions of men who belong to the society, those who do not belong
are considered in the light of those who wish to enjoy an advantage
without contributing their share to sustain it; nothing being more
certain, in the minds of the workmen, that, but for the combination, the
immediate necessities of the men would be immediately taken advantage of
in some part of the trade, which would speedily produce the same result
throughout. The predatory instinct belonging to our nature would in
these circumstances ensure this result. The men, therefore, naturally
expect that every man should pay his quota, for an advantage which he
enjoys in common with the rest. And Ave should be glad to be informed
how, in these circumstances, it is possible to be otherwise? It may be
considered---and especially by those employers who find it most
advantageous for their individual profit to deal with the men singly---
very wrong for the workmen to act thus. But it is for this very
employer's reason that the men are anxious that all should be in
combination. In fact, whatever is deemed to be the right and proper
course for the welfare of all, by the majority of a community or body of
men, and adopted by it in the aggregate, will be sure to be considered
by that community as the duty of all to support and carry out. In fact,
it is its public opinion---and how potent public opinion is need not be
repeated. And if that course involves a money payment, he who refuses or
attempts to shirk that payment will always be, let him be of what class
of society he may, ``coerced'' by its public opinion. It is so
everywhere. It may be very wrong, but if it be, men of every grade of
intellect and position in society have always done so, and, without
doubt, will continue to do so to the end of time.

But this opinion, in most places, is so universal among working men,
that in point of fact there are none to ``coerce;'' whether actually in
the society or not, scarcely a man doubts the propriety of belonging to
it. And where, as in some few districts, it is otherwise, the men do or
do not belong to the society, as they please, and consequently there is
no ``coercion.''

In our own trade those out of the society are, for the most part, either
inferior workmen, employed on inferior work at reduced wages, or those
who have belonged to it, and been erased. Of these last some left
because they did not wish to pay to it, or indeed to anything else that
they could avoid; and the rest, by far the greatest number are those who
have been erased for non-payment through their unfortunate habits of
intemperance, which left them no means of paying. The first-named, from
their inferior ability as workmen, seldom come to a society shop; the
others, when they obtain work in a society shop, generally join, or
rejoin; and, as the result of nearly twenty years' experience as
Secretary, the present writer can safely affirm that in no case have
those who have so joined, or rejoined, expressed the slightest complaint
of any ``coercion;'' on the contrary, where there has been any
expression at all, it has always been that they had been out of the
society so long, or had not joined it sooner. The practical fact is, the
chorus of assertions to the contrary notwithstanding, that the belonging
to the society is never by working men, taken generally, deemed a matter
of ``coercion'' at all, but, as a thing for the protection of their
wages from undue reduction, highly advantageous for them to do.

\subsection{UNIFORM RATE OF WAGES.}

Before discussing this topic, it is necessary to state that the
\textit{men} do not contend for an ``uniform'' rate of wages. All they
contend, or ever have contended for, is a ``\textit{minimum}'' rate of
wages, leaving the employer to pay for superior skill, or working
ability, as much more as he pleases, or the man can obtain. Indeed, as
great skill or superior working ability must, from the nature of things,
be always rare, what is termed an uniform rate can only mean what is
applicable to the general run of men, and in point of fact a
\textit{minimum} rate as regarded by the men, and an ``uniform rate'' as
regarded by the masters.

To have an ``uniform rate'' of wages is said to give an unfair
protection to the unskilled workman. The ``Quarterly'' reviewer states
it to be ``a most dangerous thing for workmen to proclaim that the idle
and unskilled shall be paid as the industrious and skilled.'' This
reviewer shows much candour throughout his remarks; he errs, however, in
this matter, from receiving his impression from only one side of the
question.

It is quite true that in most trades, where the work is paid by the day,
an uniform rate of wages is paid; but the above result, which has been
so much denounced, certainly is not, and never was, the intention of
such uniform rate or the reason why it was established.

It should here be observed that where the work in any trade is paid for
by the ``piece'' at so much per job---as, for example, among the
compositors, the type, in London, is for the most part composed at so
much per 1,000 letters---there is no uniform rate \textit{received} by
the workmen. There is generally an uniform rate, or nearly so, of the
price of the various jobs; but, as workmen of different quickness and
skill will do more or less work at the same price per job, their wages
may very materially differ in amount.

But in work by the day there is generally an uniform rate, which is
adjusted in the first instance by the principle of demand and supply,
which operates like the circulation of the blood without anyone scarcely
being aware of it, and the result is a settled rate of wages, which
becomes recognised by both employer and employed as the standard of
wages for the general run of workmen in the trade. And it must always be
borne in mind that a rate of wages, in any trade, never becomes what is
termed the uniform or standard rate, unless it has had the full
consent---very often by special agreement---of the employers of that
trade. It is quite true that some men are able to earn, and do earn,
more than others. As the uniform rate is generally made for the general
run of men, some men will be worth more and some worth less than its
amount. Then the question returns, why not pay the one more and the
other less? This is sometimes done, when, of course, the rate ceases to
be -uniform. But where the uniformity, as a general practice, is kept
up, the reason usually is---that the employer likes to reap the benefit
arising from the man being worth more, and is also afraid that if he
increased this man's wages, he would probably be called upon to raise
the others; and the men, en the other hand, would be against the man who
was not worth the regular rate taking less, lest the others might be
reduced to it. And it must be admitted that, so prone are men to take
every advantage in their own favour, the apprehension of each class
respecting the other is very likely to be correct. But it will be seen
in the sequel that this uniform rate operates in a way which is anything
but a protection to the idle and unskilled. Employers never fail to
retain in their employment the best workmen, and to discharge, at the
first opportunity, those whom they consider inferior. In the course of
time, by this process, they will have in their employ few but what may
be termed superior workmen, to whom they will pay by the operation of
this ``uniform rate'' of wages only the wages of the general run of
workmen. Digressing for a moment, as far as we can learn, this was
actually the case with the master builders, previous to their
``lock-out.'' They had in their employ the \textit{élite} as workmen of
their respective trades. Hence the great loss they have sustained by
their `` lock-out,'' which may be thus estimated. The number locked out
was under 11,000,\footnote{This is the number stated by Mr. Potter and
the Building Trades Conference. Wishing to be correct, Mr. Wales,
Secretary of the Master Builders' Association, No. 8, Great St. Helens,
was written to, requesting him, as a great favour, to state his estimate
of the total number of men locked out, skilled and unskilled. As Mr.
Wales did not reply, we have no means of collating the above number with
what that of the employers might be. However, from various
circumstances, we have good reason to believe the above number to be
correct, and our estimate of the loss very much understated.} skilled
and unskilled. The master builders, when there was reason to believe
they still wanted more hands, stated, in a return by their secretary to
the \textit{Weekly Mail}, of Nov. 13, 1859, the number in their employ
to be upwards of 16,000 (up to Nov. 5, 1859), to whom they would
certainly not be paying less per man than they were previously paying
the 11,000 for doing the same quantity of work, involving a loss, if wx
suppose the average wages of each to be 25s. per week, of upwards of
6,000 a week in wages alone. But to return. Inferior workmen, it will be
seen, by the operation of this ``uniform rate'' are not protected, its
result being to give the employer the benefit, as described, of having
superior workmen at the ordinary rate of wages. Indeed, it is very
seldom that the employers complain of this; we might almost say never,
except when it is intended to reduce wages by what might be termed
``working the uniform-rate dodge,''---that is, assuming the rate of the
general run of workmen to be that which should only be paid to the very
best workmen. This is often done when wages are sought to be reduced.
One instance, which will illustrate every other, we select from the
Parliamentary Report on Combinations of Workmen, 1838. Mr. Carolin,
master builder of Dublin, who, through his reduction of wages, was
involved in serious disputes with the workmen, after evading, by a
reference to piece work, the question put by Mr. O'Connell (7394), ``Do
you pay any man more than the minimum stipulated by the body?''
``Admitted (7348) that he paid 27s. a week to his best men and so
downwards.'' By the evidence of Mr. Eaton, also a master builder of
Dublin (7646), the regular wages of the same workmen (carpenters) were
4s. 8d. a day, or 28s. per week, which he paid, ``making no difference
between superior and inferior workmen,'' ``which he thought fair between
master and man,'' but he added, which illustrates what we have stated,
that ``he endeavoured to select the best workmen.'' The one who did not
wish to reduce wages sought and obtained the fair advantage which the
uniform rate afforded; while the other, whose only object was to reduce
wages, complained of this rate, and gave it only to his best hands. And
this is the usual course those who complain of the uniform rate pursue.
The reduction of wages is generally their object, and the uniform rate
is the mere pretence which they use to effect that object in the way
described.

Indeed, so far from placing ``the competent and the incompetent on the
same level,'' this ``uniform rate'' has been bitterly complained of, as
excluding the incompetent altogether. At the late Bradford meeting one
of the speakers gave as a reason against Trades' Unions that as he was
not able to earn the usual rate, and as the Union would not allow any of
its members to work for less, while he was a member he could get no
employment, and so he left it. In ``Chambers' Journal'' for April, 1856,
this complaint is urged with great energy; and Sir J. K. Shuttleworth,
chairman, at the discussion at Bradford on ``Trades' Unions'' in
reference to the above statement, said, ``that nothing was better
attested than that Trades' Unions did coerce their members in this
matter, and that unless they purged themselves from this taint they
would not meet with the sympathy of the public.'' Knowing as we did that
this ``uniform rate of wages,'' which, by the way, is never in any trade
without numerous exceptions, and when the work is done ``by the piece''
does not exist at all; is never in practice a source of difficulty, as
an employer is sure to dismiss the workman who does not suit him---and
we have known excellent workmen so dismissed, as not being used to the
particular kind of work the employer had to execute, though first-rate
hands in their own department; and knowing, also, that this ``uniform
rate'' is rarely complained of by employers, except they wish to reduce
wages, or by workmen unless they are arrant bunglers; we were not a
little surprised to hear of it as a matter of grave complaint. It
becomes therefore necessary to show, in respect to workmen, as we have
previously in respect to employers, how those who thus complain are
affected by this uniform rate, and what kind of workmen they are.

A man may not be worth the regular wages, either because he is slow in
doing his work, or because his work is inferior. In the first case,
where work by the piece is the custom of the trade, there is and can be
no difficulty in the matter, unless, indeed, he be so very slow that no
employer will give him room in his shop. Where, however, work in a trade
is done by the day, there may be a difficulty; still if the man's work
be well executed, he is sure, in the end to get employment where
quantity is not so much regarded in the work as its quality. But if the
man's work be inferior, there is indeed a difficulty, but that
difficulty is not in the regulation of the trade, but in the man
himself. If there were no regulation as to the rate of wages, such a man
would be restricted to those shops only where inferior work was done,
and, if his work were very inferior, even these would be closed against
him; for what master would employ a man to spoil his materials?

This kind of incompetence may proceed from imperfect instruction, in
which case the obvious course for the young man is to engage himself
as an improver; in many cases, however, that we have known, even this
has not been necessary. Where there is a natural aptitude to learn, the
man acquires more skill at every shop he works at, and ultimately
becomes as competent as the best. Where, however, the incompetence from
inferior work is the result of inability to learn, and not for the want
of proper teaching, there is no remedy; indeed, such an one has no
right to designate himself a workman of any trade, if he is not able to
work at it in the style of a workman. But here, again, the difficulty is
not in the regulation, but in the man himself. These, however, are the
men most likely to complain of this regulation. They will lay their
being continually out of employ to any cause rather than to their own
bungling execution of the work entrusted to them to do; to their not
being allowed to work for less than the standard wages; or to anything
but the real cause---their own thorough incompetence.

The reason why all men in a trade are supposed to be competent to earn
and to receive the standard rate of wages is, because, first, the fact
is, that, taken as a whole, they are all competent, the exceptions being
too few to be appreciable; and, second, because, from the predatory
instinct referred to above, which prevails equally among workmen as
among masters, some men, being thoroughly competent, would pretend
incompetence, that they might supplant their fellow workmen by offering
themselves for less than the regular wages. In our own trade, however,
where men from age or obvious inability are unable to earn the regular
wages, there is not, nor ever has been, any objection to their receiving
less.

In reference to the complaint that this uniform rate stints the action
of superior skill or physical strength, such complaint applies, wherever
it justly exists, to the employer as much as to the employed. Workmen
who are remarkably quick are sometimes paid more, but not always, as the
employer mostly likes to reap the benefit of such quickness himself;
which generally results in the quick hand subsiding into the regular
quantity of work. This is often made the ground of great complaint
against workmen that they discourage skill and celerity in their work.
The complaint, however, often should be reversed; the workman might
justly reply that he gets no encouragement. If he were, from rare
physical development, to do double the work of another, he would be paid
no more, or, if he were, it would be by no means in proportion to the
additional work he got through. This is the true reason why workmen are
accused of not wishing to domore than a certain quantity of work in a
day, about which the master builders have been so eloquent and so
denunciatory. The ``Edinburgh and Quarterly Reviews'' have in this
matter taken up their parable against the working man as advocates of
the employers. But why should the workman do more than the regular
quantity of work if he be not paid more? Would either of these two
reviewers write two sheets for the price of one? To expect them to do
more is to desire virtually a reduction of wages, which we have before
shown to be at the bottom of all these complaints of the uniform rate of
wages.

When a trade is unusually brisk, it may be that there are no hands to be
obtained but the inferior workmen, who from this cause have, until then,
remained unemployed; these receiving the usual rate does not alter the
case, as it is with them a question of simple demand and supply, which,
in these circumstances, would determine them to be worth it. Many
complaints admit of this solution, while nothing is said, when things
are exactly the reverse, of superior workmen being obtained---from the
same cause reversed---at the ordinary rate ot wages.

But, after all, we believe both employers and employed are generally
satisfied in respect to this ``uniform rate,'' as it exists in their
respective trades. During a strike, or dispute, each will say hard
things of the other, as men in such circumstances will always do. Apart
from this, if there be men, who will not do the proper quantity of work,
there are exacting employers who, if their men did double the usual
quantity of work, would grudge paying more; but these do not represent
the great body of masters or men in any trade. They are exceptions, of
whom it may justly be said, in reference to their respective
demerits---to use the vernacular---that ``there are six of one, and
half-a-dozen of the other,'' each naturally, and most surely, causing
the result of which they both complain.

But, it is asked, why cannot a man sell his labour for what he likes, as
a shopkeeper tickets his goods under the price of those of his
neighbour? They do not interfere with each other, and why should the
workman? The shopkeepers here alluded to, Ave beg to reply, are not
obliged to be always together, and, therefore, the ill blood, which is
often intense, does not cause interference. But, in wages, the matter
assumes a very different aspect, as will be seen by putting a case.
Suppose, in a shop which employed twelve or any other number of men,
whose wages were 30s. per week, a man was to be engaged who thought
proper, without being asked, to work for 25s. per week; not because he
was deficient in skill or quickness, but simply because he chose to do
so; or, if the work were done by the piece, to offer to do each job at a
corresponding reduction. Would not the rest be very likely to see, as
the result of this, a near prospect of their own wages being lowered to
this standard, and, in consequence, regard the underworker as a
``curry-favour'' and a sneak, who sought unfairly to supplant them, and
treat him accordingly? It may be very true, that this man had a right to
ask what wages he liked for his labour, even to half the above sum; but,
while men are human, Trades' Union or no Trades' Union, he would not be
able to do so without feeling his position to be highly disagreeable to
himself. We should like to know how those who put the above question
would behave if they formed the twelve, or a part of it, above supposed?

But it may be replied, this is exactly the reason why Trades' Unions are
objectionable; they unduly interfere with the natural liberty -which
such a man as this, for instance, undeniably possesses to work for what
he likes. Undoubtedly they do. But the cause of this interference is,
not in the Trades' Union, but in the natural repugnance which fill men
feel to being supplanted, and by such means either turned out of their
employment, or forced to work at a lower wage. It is this repugnance,
which is found everywhere, that prompts the ``undue interference,''
which would be felt whether the men were in combination or not.
Combination might be the effect of this feeling and make it more
effectual, but it is not its cause. Its cause is in the common instincts
of man's nature, which can never be rooted out, nor is it proper that
they should be. To suppose that it is the result of combination is to
mistake the effect for the cause, which indeed, and the assigning of a
wrong cause, are the two great mistakes made by the opponents of Trades'
Unions. The case is different when a reduction of wages proceeds from a
falling market, or from what appears to be a decline in the trade or
manufacture. The natural repugnance to such reduction here gives place
to what is seen to be inevitable. Hence Trades' Unions seldom
interfere---indeed, never---unless through ignorance of the existing
cause. This, however, can hardly happen, because, if the men were ever
so inclined, no strike can take place when there is little or no work to
do. For it will be at once seen, that men who are already unemployed
cannot ``strike'' from their work. When, in the case of the ``Luddites''
of 1812, nearly a whole trade was thrown out of employment by the sudden
introduction of machinery, the disorders which followed were not the
result of any Trades' Union combination, but were riots and violence
impelled by extreme poverty and hunger coming suddenly upon a large
number of people; which, at any time, is likely to recur under the same
circumstances, whether produced by the introduction of machinery, or any
other cause.

\subsection{PIECE WORK.}

It has been alleged that workmen refuse to work by the piece. This is
not correct. There does exist a prejudice against it by some, both of
workmen and employers, but it is not general. The compositors of London,
as a general rule, work by the piece, time-work being the exception;
while those in the country work by the day, the exception being work by
the piece. The shipwrights of the Port of London work by the job or
piece, while those of all other ports work by the day. Indeed, the main
objection in different trades to working by the piece is in the
complaint that, when the men are found to earn good wages at it, the
employer wishes to reduce the price of the work, and that it is so often
made use of, as a means of reducing wages. There are, as we are
informed, in the engineering trades three modes: first, by the piece at
a settled price for the various jobs; second, the giving out a quantity
of work, at a certain price, to one man, who gets others to do it for
less, similar to fhe sweating system among the tailors; and, third, the
selecting of a man who possesses superior physical strength and
quickness as the principal of several workmen, and paying him an
additional rate, by the quarter or otherwise, with the understanding
that he is to exert himself to the utmost to induce the others, who are
only paid the ordinary wages, to keep up to him, by which means is
obtained the work of several men, up to the standard of a workman of
superior working strength and quickness, for the payment, except in the
case of the principal, of the ordinary rate of wages. Without any
comment, this will go far to explain many of the complaints of
``\textit{stinting} the action, superior skill, and working power,''
made by the employers against their men.