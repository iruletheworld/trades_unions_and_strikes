\chapter{--- WAGES.} \label{WAGES}

It is superfluous to say that the price of labour, like that of
everything else, is determined by the quantity or supply of it
\textit{permanently} in the market; when the supply of it
\textit{permanently} much exceeds ils demand, nothing can prevent the
reduction of wages; and, conversely, when the demand for it permanently
much exceeds its supply, nothing can prevent their rise. In these two
extreme points all contention is hopeless. No Trade Society on the one
hand, however well organised, can, or ever did, prevent the fall of
wages in the first case; nor, in the other, can, or ever did, the most
stringent legislative enactment, of which there have been many
instances, prevent their rise. Trade Societies, however, rarely meddle
with these two extremes. Leaving them, we come to the intermediate
states that admit the operation of Trade Societies; and, indeed, which
call them into existence.

In all exchanges, besides the adjustment which takes place in them by
the operation of demand and supply, there is always, from the predatory
instinct inherent in the very nature of man, noticed above, a desire, on
one side or the other, to take advantage of the necessities either of
the buyer or seller; and, in proportion as these necessities are
immediate and pressing, to press that advantage accordingly. Take, for
example, an estate for sale, of which it is known that, come what will,
it must be sold in a fortnight for ready money, or what is deemed as
good as ready money. If the estate be large, the seller, under these
circumstances of necessity, will be sure to lose some thousands of
pounds. In wages, besides the rate of wages, which results from the
demand for it in proportion to its supply, there is a lower rate which
may be the result of the necessities of the workman. For example, in
those trades where there is what is called the ``Sweating System''
practised, the fair result of the demand and supply rate of wages is
represented by the amount received by the ``Sweater;''\footnote{A
``Sweafer'' is one who takes out work to do, at the usual rate of wages,
and who gets it done by others at a lower price; the difference, which
is his profit, being ``sweated'' out of those who execute the work.
Hence the term ``Sweater.''} perhaps, however, his demand, to enable him
to get the work, may be somewhat less. He, however, gets the work done
lower still; which is his profit, by employing those whose necessities
compel them to work for his price, or starve---some of these, perhaps,
may be first-rate workmen, whose habits of intemperance have reduced
them to the lowest state of destitution, and who are glad to work for
any price. There is, therefore, a wide difference between the demand and
supply rate of wages, or the rate which the fair operation of exchange
would give, if the buyer and seller of it were on equal terms, and that
which is, or would be, compelled, if the employer dealt with each man
``singly,'' and obtained the reduction which his necessities might
dictate.

The supply of labour in a trade may be greater at one time than another
--- indeed for a time far exceed the demand for it; and yet that state
not be its permanent condition. In other words the trade might be for a
good while ``slack.'' If a reduction of wages took place during this
period, this reduction would be very likely to remain when the trade got
``busy,'' the supply of hands not being then greater than the demand;
and yet the temporary ``slackness'' in itself would by no means be
adequate to compel this reduction, and ought not to have produced it.
The undue competition among employers, bidding each under the other,
intending to make up the difference out of the wages of their men,
---not from any compelling necessity in the trade, but from mere
rivalry, ---would inevitably cause a reduction, which, if not checked,
might extend to the whole; not again from any compelling necessity
arising from over-supply, but from taking advantage of the immediate
necessities of their men being greater than their own. Hence the
formation of Trade Societies, which become a necessity, to adjust the
bargain for the sale and purchase of labour.

In all bargains, the buyer wishes to buy as cheap, and the seller to
sell as dear, as he can; but their interests, all being exchangers, and
each, from the highest to the lowest, depending upon this principle
---exchange--- for his position in life and even for his daily
sustenance; all things being equal, their position is not one of
opposition, but of mutual interest; and neither the one in wishing to
get as much, nor the other in wishing to pay as little as possible, can
injure the other. But if, as alluded to above, either party possess an
advantage over the other in the bargain, this position of mutual
interest is changed into one of opposition, in which the weaker party is
sure to be deprived of some proportion of what is justly due to him. In
this position as bargainers for the sale and purchase of labour, stand
the employer and employed. Singly the employer can stand out longer in
the bargain than the journeyman; and as he who can stand out longest in
the bargain will be sure to command his own terms, the workmen combine
to put themselves on something like an equality in the bargain for the
sale of their labour with their employers. This is the
\textit{rationale} of Trade Societies, which is very clearly indicated
by Adam Smith in his ``Wealth of Nations.''